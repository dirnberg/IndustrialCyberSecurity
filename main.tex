% This template was provided for the 2019 Recording Project class at UWL.
% It is designed to provide a simple, low-configuration starter template
% for academic writing with LaTeX and Markdown.
% 
% This template is set up to use the agsm Harvard referencing style by default.

\documentclass[a4paper, 11pt]{scrartcl}
 \usepackage[top=2cm, bottom=2cm, left=2cm, right=2cm]{geometry}
 \usepackage[export]{adjustbox}

%\usepackage[german]{babel}
\usepackage[english]{babel}
\usepackage[utf8]{inputenc}
\usepackage{amsmath}
\usepackage{fancyhdr, graphicx}
\usepackage[T1]{fontenc}
\usepackage[utf8]{inputenc}
\usepackage{lmodern}
\usepackage{csquotes}
%\usepackage{natbib}
\usepackage[footnotes,definitionLists,hashEnumerators,smartEllipses,hybrid,citations]{markdown}
\usepackage{longtable}
\usepackage{color}


\usepackage{float}
\usepackage{wasysym} % für die benutzten Symbole
\usepackage[most]{tcolorbox}

\usepackage{relsize}

\usepackage[most]{tcolorbox}
\newtcolorbox{mybox}[2][]{%
  attach boxed title to top left
               = {yshift=-8pt},
  colback      = blue!5!white,
  colframe     = blue!75!black,
  fonttitle    = \bfseries,
  colbacktitle = blue!85!black,
  title        = #2,#1,
  enhanced,
}

\newtcolorbox{gebox}[2][]{%
  attach boxed title to top left
               = {yshift=-8pt},
  colback      = yellow!5!white,
  colframe     = yellow!75!black,
  fonttitle    = \bfseries,
  colbacktitle = yellow!85!black,
  title        = #2,#1,
  enhanced,
}


\newtcolorbox{nabox}[2][]{%
  attach boxed title to top left
               = {yshift=-8pt},
  colback      = red!5!white,
  colframe     = red!75!black,
  fonttitle    = \bfseries,
  colbacktitle = red!85!black,
  title        = #2,#1,
  enhanced,
}




\setlength{\parskip}{0.5em}

\newcommand{\hash}{\#} % Hashes are difficult in the Markdown environment. typing \hash within a Markdown document will give you a working hash instead.

% A basic Harvard style is the agsm style. Other similar styles are apalike and lsalike.


\bibliographystyle{ieeetr}
%\bibliographystyle{agsm}
\setkeys{Gin}{width=.95\linewidth}
\pagenumbering{gobble}% Remove page numbers (and reset to 1)

% Add your report title here:


\title{
%\vspace{-3cm}
%\hspace{10cm}
%\includegraphics[width=0.33\textwidth]{img/ikarus_logo.png}
Industrial Cyber Security}


\subtitle{Practical Guide to Secure Industrial Internet of Things}

% Add your name here:
\author{Herbert Dirnberger}

% The command below auto-generates today's date.
\date{\today}

% The \begin{} command tells LaTeX that we're starting the document.




\begin{document}


% \maketitle prints the title (as defined in the previous section) to the page.

\maketitle
% \newpage is fairly self-explanatory!
% \newpage

% \tableofcontents reads the whole document and auto-generates a table of contents based on your section headings.

\setlength{\parindent}{0pt} 

%\paragraph{Executive Summary}
%By applying this 

%\paragraph{Target Group and Scope}
%This document is primarily intended for asset owners, system integrators and service operators in all sections of the asset life cycle: Requirement Engineering, Project planning, Procurement, System Integration and System Service and Operation



\newpage
%\setcounter{tocdepth}{3}
\tableofcontents

%\subsection{% \listoffigures generates a list of images used in the document.}
%\listoffigures

\pagenumbering{arabic}% Arabic page numbers (and reset to 1)

% This is where your work goes!
% The \markdownInput{} command inserts the contents of a Markdown file into the document body. The first file 


\newpage
\markdownInput{01-enterprise-iot.md}
%\markdownInput{02-requirements.md}
%\markdownInput{99-literatur.md}
\newpage
%\markdownInput{0a-appendix.md} 
%\bibliographystyle{plainnat}
%\bibliography{references}

\end{document}
